%%
% The BIThesis Template for Bachelor Graduation Thesis
%
% 北京理工大学毕业设计(论文)第二章节 —— 使用 XeLaTeX 编译
%
% Copyright 2020-2023 BITNP
%
% This work may be distributed and/or modified under the
% conditions of the LaTeX Project Public License, either version 1.3
% of this license or (at your option) any later version.
% The latest version of this license is in
%   http://www.latex-project.org/lppl.txt
% and version 1.3 or later is part of all distributions of LaTeX
% version 2005/12/01 or later.
%
% This work has the LPPL maintenance status `maintained'.
%
% The Current Maintainer of this work is Feng Kaiyu.
%%

\chapter{基于树状区块链的出租车调度系统测试}

树状区块链将单链结构转化为树状结构的改进,提升了区块链技术在地理位置有关的应用场景下的理论效率。但截至撰写此文时,笔者尚未找到树状区块链在实际应用环境中的测试数据,难以证明其实际性能与理论性能同样优秀。本章将以实验室已有工作——基于区块链的出租车调度系统为例,使用geth1和geth-tree分别构建不同拓扑结构的区块链并于其上运行该调度系统,统计在轻压力负载和中等压力负载下司机侧和乘客侧各关键节点的时间戳并将结果可视化,以测试树状区块链在实际应用场景中的性能表现。同时,重构已有的脚本代码,增强其可扩展性和易用性,方面后人进行本测试实验的复现。

\section{设计测试}

本测试分为基于树状区块链geth-tree的测试和基于区域索引区块链geth1的测试两部分,用以对比两种区块链实现在同一套系统下的性能表现差异。

两部分实验中均在真实世界地图中Geohash编码前缀为wx4e的区域下进行。树状区块链部分的实验在该区域下的细分区域wx4en、wx4ep、wx4eq和wx4er区域下进行。每个区域中,均存在16位司机和32位乘客,所有司机的初始位置均相同,所有乘客的出发地点和目的地也相同。以上地点的选点工作基于蒙思洁完成的真实地图信息提取与筛选工作进行,已提前确保选择的路线可以在真实世界地图上导航成功。

本测试使用JavaScript脚本模拟司乘交互行为。司机模拟脚本负责读取司机的公钥地址、初始位置,并将其上传到链上;随后,司机将监听一系列合约事件,例如乘车请求事件、乘客付款事件等,并作出响应。乘客模拟脚本负责读取乘客的公钥地址、起始点位置和目的地位置,并将其上传到链上;每隔一段时间(在树状区块链中为3秒,在区域索引区块链中为前者的$\frac14$),将有一名乘客发射乘车请求事件。发射乘车请求事件后乘客将在临近区域周边搜索与自身曼哈顿距离最近的车辆并尝试占有。若车辆已被占用,则等待一段时间后再次重复上述步骤,直至车辆分配成功。上述等待时间的计算规则为$t_{waiting} = 10 + 4 \times $重试次数。

进行树状区块链部分的实验时,首先搭建树状结构,在每个子链上分别部署合约。待合约部署完毕后,所有子链并行地运行模拟司乘交互行为的JavaScript脚本,该脚本读取对应子链管辖的细分区域内的司乘信息,并记录司乘双方各自在调度过程关键节点的时间戳。

在区域索引区块链测试部分中,测试步骤大致相同,但运行司乘交互模拟脚本时,应令其读取所有四个细分区域内所司乘的信息,模拟不进行区域细分,使用单链结构区块链运行出租车调度系统的应用场景。

\section{测试环境}

基于树状区块链的调度系统测试在如下环境中进行。由于运行四子链的性能开销较大,笔者放弃了使用虚拟机,而使用Windows下的Linux子系统(WSL 2),以提升开发体验。

\begin{table}[htbp]
    \linespread{1.5}
    \zihao{5}
    \centering
    \caption{树状区块链调度系统测试环境}\label{树状区块链调度系统测试环境}
    \begin{tabular}{r|l} \toprule
        中央处理器 & Intel Core i5-12500H      \\
        图形处理器 & Intel Iris Xe 80EU        \\
        内存    & 24GB                      \\
        操作系统  & Ubuntu 22.04.2 LTS        \\
        虚拟机   & Windows Subsystem Linus 2 \\
        \bottomrule
    \end{tabular}
\end{table}

\section{测试步骤}

本节的详细测试步骤已记录于在线代码仓库\footnote{\url{https://gitcode.net/qq_39710999/taxi-4-leaves}},故本节仅简要介绍大致测试方法。

\begin{enumerate}
    \item 建立有四个子链的树状区块链网络,确保四个子链中均拥有相同的192个账号,并已经为它们执行解锁操作;
    \item 在四子链上分别部署合约,并使用得到的合约地址更新各脚本中存储的合约地址,随后上传地图;
    \item 依次启动子链挖矿,注意分配CPU核心数量相同以控制变量,随后启动模拟司机与乘客交互行为的脚本,等待测试结束;
    \item 在仓库根目录将生成结果报告。
\end{enumerate}

\section{测试数据分析}

\section{本章小结}
