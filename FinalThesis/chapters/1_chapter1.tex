%%
% The BIThesis Template for Bachelor Graduation Thesis
%
% 北京理工大学毕业设计(论文)第一章节 —— 使用 XeLaTeX 编译
%
% Copyright 2020-2023 BITNP
%
% This work may be distributed and/or modified under the
% conditions of the LaTeX Project Public License, either version 1.3
% of this license or (at your option) any later version.
% The latest version of this license is in
%   http://www.latex-project.org/lppl.txt
% and version 1.3 or later is part of all distributions of LaTeX
% version 2005/12/01 or later.
%
% This work has the LPPL maintenance status `maintained'.
%
% The Current Maintainer of this work is Feng Kaiyu.
%
% 第一章节

\chapter{绪论}

\section{研究背景}

2008年,一位自称为中本聪的人发布了名为《Bitcoin: A Peer-to-Peer Electronic Cash System》的论文,宣告了区块链技术的诞生。区块链,乃是一个分布式的账本;区块链网络不存在所谓的“中心服务器”,每台参与构成区块链网络的计算机(又被称为“节点”)均持有一份该账本的副本。通过称为共识算法的机制,各节点能够就区块链的当前状态达成一致,并在链上数据发生变化时及时追踪并更新到自身存储的账本中;不仅如此,若某个节点尝试擅自修改自身所持有的账本,其行为会被共识算法拒绝,从而规避了恶意篡改链上数据的风险。上述区块链的优势,令区块链这一新兴的概念迅速为各行各业接受:中国人民银行数字货币研究所正在积极探索区块链技术在低并发、低敏感的资产确权、交易转让、账本核对等场景下的应用\cite{shuYanSuo};区块链透明化的特点和极高的安全性也引起了地产行业的注意\cite{usageOfBC}。可以预见,区块链技术在未来将吸引更多行业加入,以其去中心化、不可篡改等特性造福人类社会。

车联网技术(Internet on Vehicle)是物联网技术的子集

并迅速以其去中心化、不可篡改等特点,迅速博得了大量行业的青睐。目前,区块链技术已经在数字货币、车联网等领域取得了亮眼的成绩。然而,传统区块链采用单链结构,每个区块仅存储上一个区块的哈希信息,在需要应对大吞吐量的工况下,容易因链条过长导致性能下降;此外,在应用于车联网这一场景下时,由于区块并未按照地理位置存储,而车辆节点需要关心的信息大多来自临近区域的区块数据,故可能需要耗费诸多不必要的查询开销。

为解决上述两个问题,“树状区块链”应运而生。在树状区块链中,区块被分为分支区块和叶子区块两种。叶子区块和传统的区块链并无太大差异,而分支区块则负责将数个叶子区块组织起来,按照叶子区块所代表的地理位置,结合GeoHash编码技术形成类似于字典树的树状结构。由于树状结构相比单链结构的深度更小,且采用了与地理位置相关的GeoHash进行分支构造,故有望为上述两个问题提供合理的解决方案。

\section{二级题目}
% 这里插入一个参考文献,仅作参考

\subsection{三级题目}

正文……\cite{yuFeiJiZongTiDuoXueKeSheJiYouHuaDeXianZhuangYuFaZhanFangXiang2008}……\cite{Hajela2012Application}

\textcolor{blue}{正文部分:宋体、小四;正文行距:22磅;间距段前段后均为0行。阅后删除此段。}

\textcolor{blue}{图、表居中,图注标在图下方,表头标在表上方,宋体、五号、居中,1.25倍行距,间距段前段后均为0行,图表与上下文之间各空一行。阅后删除此段。}

\textcolor{blue}{\underline{\underline{图-示例:(阅后删除此段)}}}


\begin{figure}[htbp]
  \centering
  \includegraphics[]{images/bit_logo.png}
  \caption{标题序号}\label{标题序号} % label 用来在文中索引
\end{figure}

\textcolor{blue}{\underline{\underline{表-示例:(阅后删除此段)}}}
% 三线表
\begin{table}[htbp]
  \linespread{1.5}
  \zihao{5}
  \centering
  \caption{统计表}\label{统计表}
  \begin{tabular}{*{5}{>{\centering\arraybackslash}p{2cm}}} \toprule
    项目    & 产量    & 销量    & 产值   & 比重    \\ \hline
    手机    & 1000  & 10000 & 500  & 50\%  \\
    计算机   & 5500  & 5000  & 220  & 22\%  \\
    笔记本电脑 & 1100  & 1000  & 280  & 28\%  \\
    合计    & 17600 & 16000 & 1000 & 100\% \\ \bottomrule
    \end{tabular}
\end{table}

\textcolor{blue}{公式标注应于该公式所在行的最右侧。对于较长的公式只可在符号处(+、-、*、/、$\leqslant$ $\geqslant$ 等)转行。在文中引用公式时,在标号前加“式”,如式(1-2)。阅后删除此
段。}

\textcolor{blue}{公式-示例:(阅后删除此段)}
% 公式上下不要空行,置于同一个段落下即可,否则上下距离会出现高度不一致的问题
\begin{equation}
    LRI=1\ ∕\ \sqrt{1+{\left(\frac{{\mu }_{R}}{{\mu }_{s}}\right)}^{2}{\left(\frac{{\delta }_{R}}{{\delta }_{s}}\right)}^{2}}
\end{equation}

\subsubsection{生僻字}

% 一个可能无法正常显示的生僻字
% 一个可能无法正常显示的生僻字: 彧。下文注释中,介绍了如何通过自定义字体来显示生僻字。

% 定义一个提供了生僻字的字体,注意要确保你的系统存在该字体
% \setCJKfamilyfont{custom-font}{Noto Serif CJK SC}

% 使用自己定义的字体
% 使用提供了相应字型的字体:\CJKfamily{custom-font}{彧}。

