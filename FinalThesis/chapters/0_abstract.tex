%%
% The BIThesis Template for Bachelor Graduation Thesis
%
% 北京理工大学毕业设计(论文)中英文摘要 —— 使用 XeLaTeX 编译
%
% Copyright 2020-2023 BITNP
%
% This work may be distributed and/or modified under the
% conditions of the LaTeX Project Public License, either version 1.3
% of this license or (at your option) any later version.
% The latest version of this license is in
%   http://www.latex-project.org/lppl.txt
% and version 1.3 or later is part of all distributions of LaTeX
% version 2005/12/01 or later.
%
% This work has the LPPL maintenance status `maintained'.
%
% The Current Maintainer of this work is Feng Kaiyu.

% 中英文摘要章节
\begin{abstract}
% 中文摘要正文从这里开始
2008年,随着中本聪发布《Bitcoin: A Peer-to-Peer Electronic Cash System》论文,区块链技术横空出世,并迅速以其去中心化、不可篡改等特点,迅速博得了大量行业的青睐。目前,区块链技术已经在数字货币、车联网等领域取得了亮眼的成绩。然而,传统区块链采用单链结构,每个区块仅存储上一个区块的哈希信息,在需要应对大吞吐量的工况下,容易因链条过长导致性能下降;此外,在应用于车联网这一场景下时,由于区块并未按照地理位置存储,而车辆节点需要关心的信息大多来自临近区域的区块数据,故可能需要耗费诸多不必要的查询开销。

为解决上述两个问题,“树状区块链”应运而生。在树状区块链中,区块被分为创世块、分支区块和叶子区块三种。叶子区块和传统的区块链并无太大差异;分支区块则负责将数个叶子区块组织起来,按照叶子区块所代表的地理位置,结合GeoHash编码技术形成多叉树树结构;创世块和分支区块类似,但它没有父链指针。由于树状结构相比单链结构的深度更小,且采用了与地理位置相关的GeoHash进行分支构造,故树状区块链有望为上述两个问题提供合理的解决方案。

本课题首先研究树状区块链相较传统单链结构区块链的优势与局限性;其次,对实验室已有工作——基于区块链的出租车调度系统进行复现,验证其可用性;对于树状区块链引入的特色功能——跨子链转账,设计系统的性能测试,并建立简单的数学模型,对开发者在不同应用场景下对树状区块链和传统单链结构区块链的选择提供建议。在上述工作的基础上,以出租车调度系统为背景,测试树状区块链和传统链式区块链在运行该调度系统时的性能表现差异。最后,提出一种使用Rust编程语言重写树状区块链的可行方案,讨论将现有树状区块链的开发平台由以太坊开发平台迁移至更加优秀Substrate开发框架的优势及可行性,并进行部分树状区块链的功能特性的重写工作加以佐证。

\end{abstract}

% 英文摘要章节
\begin{abstractEn}
% 英文摘要正文从这里开始
In 2008, with the release of Satoshi Nakamoto's "Bitcoin: A Peer-to-Peer Electronic Cash System" paper, blockchain technology came into the world, and quickly with its decentralized, tamper-free characteristics, quickly won the favor of a large number of industries. At present, blockchain technology has made remarkable achievements in digital currency, Internet of vehicles and other fields. However, the traditional blockchain adopts the single-chain structure, and each block only stores the hash information of the previous block. Under the working condition of large throughput, the performance is likely to be degraded due to the long chain. In addition, when applied to the scenario of the Internet of vehicles, because the block is not stored according to the geographical location, and most of the information that the vehicle node needs to care about is from the block data in the adjacent area, it may consume a lot of unnecessary query costs.

To solve the above two problems, "tree-like blockchain" came into being. In a tree-like blockchain, blocks are divided into genesis blocks, branch blocks and leaf blocks. The leaf block is not very different from the traditional blockchain, while the branch block is responsible for organizing several leaf blocks, according to the geographical location of the leaf block, combined with GeoHash coding technology to form a tree structure similar to dictionary tree. The major difference between the branch block and the genesis block is that a genesis block does not have the so-called "parent block pointer". Due to the smaller depth of the tree structure compared to the single chain structure and the use of GeoHash for branch construction, it is expected to provide a reasonable solution to the above two problems.

This project first studies the advantages and limitations of tree-like blockchain compared to traditional single-chain structured blockchain; secondly, it reproduces the existing work in the laboratory - a taxi dispatch system based on blockchain, to verify its usability; for the special function introduced by the tree-like blockchain - cross-sub-chain transfer, design system performance testing, and establish a simple mathematical model to provide suggestions for developers to choose between tree-like blockchain and traditional single-chain structured blockchain in different application scenarios. On the basis of the above work, taking the taxi dispatch system as the background, test the performance difference between tree-like blockchain and traditional chain-like blockchain when running this dispatch system. Finally, propose a feasible plan to rewrite tree-like blockchain using Rust programming language, discuss the advantages and feasibility of migrating the existing tree-like blockchain development platform from Ethereum development platform to a better Substrate development framework, and partially rewrite some functional features of tree-like blockchain to provide evidence.

\end{abstractEn}
