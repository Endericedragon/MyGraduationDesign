%%
% The BIThesis Template for Bachelor Graduation Thesis
%
% 北京理工大学毕业设计(论文)第二章节 —— 使用 XeLaTeX 编译
%
% Copyright 2020-2023 BITNP
%
% This work may be distributed and/or modified under the
% conditions of the LaTeX Project Public License, either version 1.3
% of this license or (at your option) any later version.
% The latest version of this license is in
%   http://www.latex-project.org/lppl.txt
% and version 1.3 or later is part of all distributions of LaTeX
% version 2005/12/01 or later.
%
% This work has the LPPL maintenance status `maintained'.
%
% The Current Maintainer of this work is Feng Kaiyu.
%%

\chapter{改进树状区块链——从以太坊到Substrate}

树状区块链是在以太坊官方客户端Go-Ethereum的源代码上修改而来,因此,虽然在结构上做出了很大的调整,它也继承了许多Go-Ethereum的特点,例如共识算法和EVM虚拟机等特点,其性能表现也依然受制于Go-Ethereum。从整体上评估,第三章的3.5.2.1节通过统计学方法,验证了以太坊顺序串行执行交易的特点,这样的执行策略使得以太坊在面对高并发请求时的处理效率不尽如人意;从局部评估,研究\cite{privateChainConsensus}表明,以太坊所使用的共识算法之一——基于工作量的证明(Proof of Work),其性能表现已落后其他更先进的算法。然而,以太坊并未在源代码层面留有太多的可扩展空间,这也意味着许多诸如更换共识算法,修改交易执行逻辑等的自定义修改在实践时困难重重,限制了在以太坊平台改良优化的空间。

Substrate\cite{substrateHome}由Parity Technologies推出,是一套开源的区块链开发框架,允许开发者针对不同的用途对链进行不同程度的定制。在Substrate诞生前,人们花费了大量的精力,试图设计一个支持多链结构的新型区块链。然而,所有这些花费的时间、金钱和精力最终导向了一个结论:当下做出的深思熟虑的选择很可能成为未来的绊脚石。这是因为随着时间的推移,区块链依赖的某些特定的技术或假设,可能会阻碍并最终扼杀创新\cite{substrateDoc}。因此,以太坊创始人之一Gavin Wood成立了Parity技术公司,力图改写这一局面。他们的处女座——以太坊客户端Parity,在相同的硬件配置环境下展现出了远胜Go-Ethereum的性能表现,提升幅度达到了可观的89.8\%\cite{parityVSgeth};在后续开发Parity自研的区块链Polkadot时,Gavin意识到,仅需将Polkadot进行抽象,剥离部分细节,即能获得一个可扩展性极强,适用范围更广的区块链框架。在2018年,Polkadot和用于开发它的区块链框架终于被分离开,成为两个独立的项目,而后者,即是本章讨论的主角——Substrate。

Substrate在设计时,严格遵循三点原则:

\begin{itemize}
    \item 将Rust编程语言作为代码库的核心编程语言。虽然Rust语言的学习曲线较为陡峭,但其极快的速度,极具辨识度的内存管理方式,灵活的抽象能力,以及可编译为WebAssembly的特点使它成为需要高性能表现,强内存安全性,及嵌入式设备友好性等特性之应用场景的不二之选;
    \item 将WebAssembly作为应用程序逻辑的执行环境。WebAssembly是一种新型代码,由万维网联盟创建,可从Rust、C、C++等语言编译获得,且受到多种JavaScript引擎的广泛支持,具有良好的兼容性\cite{wasmIntro}。Substrate的易升级性也建立于WebAssembly的基础之上:它将区块链的具体业务逻辑编译为WebAssembly字节码,并存储于区块链的数据存储区中,用户可以像发起普通交易一样发起一个申请修改链上存储的WebAssembly字节码的交易,从而便利地更新升级区块链系统;
    \item 广泛使用分层抽象、泛型实现和灵活的API作为主要的编码实践,并将库分离为不同的体系结构组件。在核心功能方面,Substrate官方提供了许多不同的实现,例如数据库层的RocksDB和ParityDB,共识层的AURA引擎和Grandpa引擎等,可以任由开发者选择;在应用功能方面,Substrate允许开发者调用官方已开发妥当的模块pallet为他们的区块链添加自定义功能,例如保存并处理账号信息的balances模块,和管理智能合约的contracts模块;不仅如此,Substrate也提供了这些模块的实现源代码,开发者可以自行下载并进行修改后引入区块链,实现功能的定制化。这一设计原则,赋予了Substrate极好的可扩展性,便于开发人员依据实际需要进行功能增删和优化改进等操作。
\end{itemize}

综合以上事实,将树状区块链自以太坊开发平台迁移至Substrate开发框架内,不仅能降低开发难度,获得更好的性能表现和安全性,还能获得更好的兼容性,令区块链能够在浏览器中乃至嵌入式设备上运行,拓宽树状区块链的应用范围。本章首先分析Substrate开发框架的架构,其次以官方提供的节点模板为例介绍其代码结构,最后在节点模板的基础上,引入树状区块链的部分特性——账号地理位置,以证明将树状区块链从以太坊开发平台迁移至Substrate开发框架的可行性。

\section{Substrate框架的架构}

\section{Substrate节点模板的代码结构}

\section{为账户加入地理位置属性}

\section{本章小结}
