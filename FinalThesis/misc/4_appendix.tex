%%
% The BIThesis Template for Bachelor Graduation Thesis
%
% 北京理工大学毕业设计(论文)附录 —— 使用 XeLaTeX 编译
%
% Copyright 2020-2023 BITNP
%
% This work may be distributed and/or modified under the
% conditions of the LaTeX Project Public License, either version 1.3
% of this license or (at your option) any later version.
% The latest version of this license is in
%   http://www.latex-project.org/lppl.txt
% and version 1.3 or later is part of all distributions of LaTeX
% version 2005/12/01 or later.
%
% This work has the LPPL maintenance status `maintained'.
%
% The Current Maintainer of this work is Feng Kaiyu.
%
% Compile with: xelatex -> biber -> xelatex -> xelatex

\begin{appendices}
    % 这里示范一下添加多个附录的方法:
    % 使用 \section 来添加一个附录

    \section{创世配置文件}

    \begin{lstlisting}[caption={创世配置文件}, label={lst:创世配置文件}]
    {
        "config": {
            "chainId": 666,
            "homesteadBlock": 0,
            "eip150Block": 0,
            "eip150Hash": "0x0000000000000000000000000000000000000000000000000000000000000000",
            "eip155Block": 0,
            "eip158Block": 0,
            "byzantiumBlock": 0,
            "constantinopleBlock": 0,
            "petersburgBlock": 0,
            "istanbulBlock": 0,
            "ethash": {}
        },
        "nonce": "0x0",
        "timestamp": "0x5ddf8f3e",
        "extraData": "0x0000000000000000000000000000000000000000000000000000000000000000",
        "gasLimit": "0xffffffff",
        "difficulty": "0x20000",
        "mixHash": "0x0000000000000000000000000000000000000000000000000000000000000000",
        "coinbase": "0x0000000000000000000000000000000000000000",
        "alloc": {},
        "number": "0x0",
        "gasUsed": "0x0",
        "parentHash": "0x0000000000000000000000000000000000000000000000000000000000000000"
    }
    \end{lstlisting}

    一些关键字段的解释如下:

    \begin{itemize}
        \item gasLimit:区块链为防止恶意参与者不停发送交易耗尽服务器资源,往往都对交易进行“收费”。gasLimit字段限制一次交易的最大花费。为保证实验成功,故此处设置得较大。
        \item difficulty:挖矿难度。难度越低,越容易挖到符合要求的新区块,出块速度也越高。
        \item alloc:记录链上部分账户的余额等信息。由于该创世配置文件乃是为尚未准备账户的全新区块链所准备,故此项为一个空白的JavaScript对象。
    \end{itemize}

    \section{合约部署的代码模板}

    \begin{lstlisting}[caption={合约部署代码模板}, label={lst:合约部署代码模板}]
    abi = JSON.parse("经过压缩转义后的ABI")
    bytecode = "获得的字节码字符串"

    Contract = web3.eth.contract(abi);
    web3.eth.estimateGas({data: bytecode})
    contractInstance = Contract.new({
        from: web3.eth.accounts[0],
        data: bytecode,
        gas: '3000000',
        position:"w2511111111111",
        txtime:277001
      },function (e, contract){
        console.log(e, contract);
        if(!e){
            if(!contract.address) {
                console.log("Contract transaction send: TransactionHash: " + contract.transactionHash + " waiting to be mined...");
            } else {
                console.log("Contract mined! Address: " + contract.address);
                console.log(contract);
            }
        }
    });
    \end{lstlisting}

    \section{跨链转账测试的数据可视化代码}
    \begin{lstlisting}[
    caption={跨链转账测试的数据可视化},
    label={lst:跨链转账测试的数据可视化},
    morekeywords={import, from, for, in, if, else, and, or, not, with, as}
    ]
    import matplotlib.pyplot as plt
    from matplotlib.axes._axes import Axes
    from matplotlib.figure import Figure
    import os

    plt.style.use('_mpl-gallery')

    dir_prefix: str = input("Directory name: ")

    candidates: list[str] = list()
    for each in os.listdir():
        if os.path.isdir(each) and each.startswith(dir_prefix):
            candidates.append(each)

    if len(candidates) > 1:
        print("Choose one:")
        for (i, each) in enumerate(candidates):
            print(f"({i}) {each}")

        choice: int = int(input("Choice: "))
        dir_name: str = candidates[choice]
    else:
        dir_name: str = candidates[0]

    # make data
    with open(os.path.join(dir_name, "tx_request_w12.txt"), "r", encoding="utf-8") as f:
        start_time = int(f.readline().split('\t')[1])

    times: list[int] = [start_time]
    with open(os.path.join(dir_name, "tx_result_w11.txt"), "r", encoding="utf-8") as f:
        times += [int(each.split('\t')[1]) for each in f.readlines()]
    times.sort()

    y: list[int] = list(filter(lambda x: x != 0 and x < 55, [times[i] - times[i - 1] for i in range(1, len(times))]))
    x: range = range(len(y))

    # plot
    fig, ax = plt.subplots()
    fig: Figure
    ax: Axes

    stem = ax.stem(x, y)

    for (i, j) in zip(x, y):
        plt.text(i, j + 0.3, str(j))

    print(f"total time consume = {times[-1] - times[0]}")

    print(f"fastest = {min(y)}")
    print(f"variance = {max(y)}")

    average: float = sum(y) / len(y)
    print(f"average = {average:.04f}")

    variance: float = sum((average - each) ** 2 for each in y) / len(y)
    print(f"variance = {variance:.04f}")

    plt.show()

  \end{lstlisting}
\end{appendices}
