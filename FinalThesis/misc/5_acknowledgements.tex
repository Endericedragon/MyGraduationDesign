%%
% The BIThesis Template for Bachelor Graduation Thesis
%
% 北京理工大学毕业设计(论文)致谢 —— 使用 XeLaTeX 编译
%
% Copyright 2020-2023 BITNP
%
% This work may be distributed and/or modified under the
% conditions of the LaTeX Project Public License, either version 1.3
% of this license or (at your option) any later version.
% The latest version of this license is in
%   http://www.latex-project.org/lppl.txt
% and version 1.3 or later is part of all distributions of LaTeX
% version 2005/12/01 or later.
%
% This work has the LPPL maintenance status `maintained'.
%
% The Current Maintainer of this work is Feng Kaiyu.
%
% Compile with: xelatex -> biber -> xelatex -> xelatex

% 致谢部分尽量不使用 \subsection 二级标题,只使用 \section 一级标题
\begin{acknowledgements}
  我曾幻想,当我完成毕业设计,心中会作如何感想呢?是完成一项挑战的成就感?抑或是将知识应用于实践的喜悦?然而,放下执笔之手,心中却只有了却一桩大事的释然。回想那遥远的秋天,怀揣好奇与些许不安踏入北京理工大学的校门,拘谨地与未来四年的同窗们打招呼的情形仿佛就在昨日一般。一路走来,虽有新冠疫情侵扰、求学之路上亦非一帆风顺,但我十分庆幸高三并非我知识储备与智力的巅峰。四年求学之路,终有所收获。

  我想感谢我的导师陆慧梅老师以及向勇老师。在我迷茫低落之际,陆老师总能以淳淳教诲重新唤起我的信心,坚定我的目标,令我能够走出迷雾,继续前行。在课题方向选择、研究思路方面,向老师给了我相当的支持,并以严格而非严苛的要求因材施教,敦促我在个人能力范围内保质保量地完成毕业设计。

  我想感谢实验室的成佳壮学长、万琦玲学姐、周畅学姐。在有关出租车调度系统和树状区块链的使用、测试、调试等过程中,他们提供了宝贵的帮助。万学姐的复现手册、成学长和周学姐组织的线上研讨会,无不令我在研究课题的过程中减少一份疑惑,增添一份信心。

  感谢我的父母傅盛阳先生和赵晓斐女士,他们在我最迷茫的低谷期毅然决然地选择无条件的理解和包容,为我出谋划策,安定我心。我的毕设能完成至如此程度,离不开他们的大力支持。

  感谢所有老师,和与我一同度过四年美好时光的同学们。有了他们,我的求学之路上,就有了朗朗书声,欢歌笑语相伴。
\end{acknowledgements}
